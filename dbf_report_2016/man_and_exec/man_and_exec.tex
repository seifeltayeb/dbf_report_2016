\section{Executive Summary}

%%%%% THE MISSIONS AND AIRCRAFT
The objective of this report is to thoroughly describe the design, manufacture and testing the University of Glasgow AIAA Design, Build and Fly team has conducted in order to create a competitive and compliant aircraft. The aim for this year's competition is to produce two radio controlled aircraft: the Production Aircraft (PA) and the Manufacturing Support Aircraft(MSA). The MSA must be able to conduct two missions, the first without a payload and the second with its payload. For the first mission the MSA must be able to fly three laps within a five minute window. After completing the first mission, the MSA must takeoff with the PA sub-assemblies carried internally then fly one lap, return and land then taxi to a designated payload change area. The procedure is to be repeated as many times as needed as long as it is done within a ten minute window. For the third mission the PA must be able to fly three laps within five minutes with a 32 oz (0.9 \si{\kilo\gram}) Gatorade bottle payload. For the bonus ground mission the ground crew must assemble the PA from its sub-assemblies and install its payload within 2 minutes.

\nomenclature{PA}{Production Aircraft}% 
\nomenclature{MSA}{Manufacturing Support Aircraft}%

%%%%% SCORING
The total score for the team is obtained from three components: \textit{Total Mission Score}, \textit{Written Report Score} and \textit{Rated Aircraft Cost} (RAC). The total mission score is a multiplication of all flight mission scores plus the ground mission bonus. RAC includes the MSA and PA empty weights, the MSA and PA battery weights and the PA's number of subassemblies. Restrictions include a maximum takeoff distance of 100 ft. (30 \si{\meter}) .

\nomenclature{RAC}{Rated Aircraft Cost} 

From these requirements it was determined through competition score analysis that empty weight and number of sub-assemblies were the two driving factors in the design, followed by the number of required flights, the weight ratios of the subassemblies and the battery weights . For example, if the team goes from having one sub-assembly to two sub-assemblies, the total mission score will go down by XX\%.

%%%%% EVALUATION AND REASONING
It was evaluated that the team could not carry the PA internally as only one sub-assembly due to sizing and manufacturing constraints. This makes mission 2,the MSA delivery flight,


%%%%% SUMMARY OF THE AIRCRAFT
The MSA's final design was a XX \si{\kilo\gram} empty weight highwing monoplane with maximum flight speed of XX \si{\meter\per\second} . The main construction materials were carbon fiber and foam.  The aircraft is able to complete 3 laps in XXXX minutes for manufacturing flight 1 (MF1) and can transport all the PA's sub-assemblies in manufacturing flight 2 (MF2) in a total of XX minutes and XX flights. 

The PA's final design was a XX \si{\kilo\gram} empty weight highwing monoplane with maximum flight speed of XX \si{\meter\per\second} . The main construction materials were carbon fiber and foam.  The aircraft is able to complete 3 laps in XXXX minutes for the production flight (PF).


%%%%%%%%%%%%%%%%%%%%%%%%
%%%%%%%%%%%%%%%%%%%%%%%% MANAGEMENT SUMMARY
%%%%%%%%%%%%%%%%%%%%%%%%

\section{Management Summary}

\subsection{Team Organization}

The University of Glasgow has team members from first year undergraduates to postgraduate students. Overall, the team had 23 members from XX countries, supervised by a faculty advisor. The project was entirely led by students who joined the project as an extra-curricular activity. Since this was the second entry for the University of Glasgow, the management team had to expand the team's capabilities by acquiring additional tools, introducing more advanced manufacturing techniques and seeking sponsors to obtain the required funding. The organizational structure of the team is shown in Figure \ref{fig:org_chart}

\begin{figure}[H]
    \centering
    \includegraphics[width=9.72cm]{./preliminary_design/fig/dummy}
    \caption{MSA drag polar}
    \label{fig:org_chart}
\end{figure}

In order to coordinate the members' efforts, the team met weekly to discuss new ideas, rectify existing issues and plan for the week ahead. Moreover, Slack -a cloud-based messaging and collaboration tool- was used in conjunction with Google Drive in order to enhance off-campus communication. The project manager was responsible for setting up the master schedule, set overall project goals and supervised the team's external communications. The operations manager created a detailed schedule that follows the master plan, obtained necessary approvals and coordinated the sub-teams. The chief engineer supervised all design processes, validated the product and reported to the project and operations managers.

\subsection{Milestones Chart}

The project manager along with the operations manager and chief engineer created a general schedule for the project. The chief engineer identified the necessary milestones 

%Schedule / Milestone chart
\begin{figure}[H]
    \centering
        \noindent\resizebox{\textwidth}{!}{
        \begin{tikzpicture}[x=.25cm, y=1cm]
            % \begin{ganttchart}[
            % vgrid={*1{draw=dbfblue!50, line width=.5pt}},            canvas/.append style={fill=none, draw=black!5, line width=.5pt},
            % bar/.append style={fill=dbfdark},
            % group peaks height=.25,
            % y unit title=.6cm,
            % y unit chart=0.6cm,
            % x unit=0.2cm,
            % group label font={\bfseries\color{dbfdark}},
            % milestone label font=\color{dbfdark},
            % milestone/ .append style={fill=dbfblue},
            % bar label font=\color{dbfblue}]{1}{28}
            % \gantttitle{2015}{14}
            % \gantttitle{2016}{14}\\
            % \gantttitlelist{40,...,53}{1}
            % \gantttitlelist{1,...,14}{1}\\
            % \ganttgroup{Conceptual Design}{1}{10} \\
            %     \ganttbar{Mission Constraint Analysis}{1}{3} \\
            %     \ganttbar{Configuration and Formulation}{4}{10} \\
            %         \ganttmilestone{3 Concepts are Formulated}{11}\\
            % \ganttgroup{Preliminary Design}{11}{20} \\
            %     \ganttbar{Mission Constraint Analysis}{20}{3} \\
            %     \ganttbar{Configuration and Formulation}{4}{10} \\
            % \ganttgroup{Detailed Design}{21}{28} \\
            % \end{ganttchart} \\
            
 
\definecolor{dbfdark}{HTML}{13375E}
\definecolor{dbfblue}{HTML}{0498D6}

\ganttset{%
calendar week text={%
\currentweek%
}%
}

\begin{ganttchart}[
vgrid={*{6}{draw=none}, dotted},
x unit=.08cm,
y unit title=.6cm,
y unit chart=.5cm,
milestone/.append style={fill=gray, rounded corners=3pt},
group peaks height=1,
group peaks width=1.5,
group/.append style={draw=none,fill=dbfblue},
group label font={\bfseries\color{dbfblue}},
milestone label font = {\color{gray}},
bar/.append style={draw=none,fill=dbfdark},
bar label font={\color{dbfdark}},
today=2015-12-01,
today offset=.5,
today label = Proposal Submission,
time slot format=isodate,
time slot format/start date=2014-12-01]{2015-09-01}{2016-04-15}
\ganttset{bar height=.5}
\gantttitlecalendar{year, month=shortname, week} \\ 

\ganttgroup{Recruitment and Planning}{2015-09-01}{2015-10-01}\\

%%%%%%%%%%%%
\ganttgroup{Conceptual Design}{2015-10-01}{2015-10-15}\\

%%%%%%%%%%%%
\ganttgroup{Preliminary Design}{2015-10-16}{2015-11-15}\\

\ganttgroup{Detailed Design}{2015-11-15}{2016-01-15}\\
\ganttbar{Material and Component Testing}{2015-11-16}{2015-12-19}\\
\ganttbar{Propulsion Testing}{2015-12-20}{2016-01-15}\\

\ganttgroup{Manufacturing and Testing}{2016-01-15}{2016-04-15}\\
\ganttbar{Prototype 1}{2016-01-16}{2016-01-30}\\
\ganttbar{Prototype 2}{2016-01-30}{2016-02-20}\\
\ganttbar{Competition Aircrat}{2016-02-21}{2016-04-04}\\

\ganttmilestone{Design Report}{2016-02-22}

\end{ganttchart}
        \end{tikzpicture}
        }
    \label{fig:gantt}
    \caption{Gantt Chart}
\end{figure}   


\begin{description}
    \item [Aerodynamics] airframe design, stability analysis, performance specification and estimation
    \item [Mechatronics] designing and testing actuation mechanisms, actuation surfaces
    \item [Propulsion] optimizing weight and power, collecting data and choosing optimum motor-propeller combinations for missions
    \item [Structures] evaluating potentially usable materials, testing obtained material,structural analysis of design, weight-optimization.
\end{description}